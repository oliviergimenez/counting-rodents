% Options for packages loaded elsewhere
\PassOptionsToPackage{unicode}{hyperref}
\PassOptionsToPackage{hyphens}{url}
\PassOptionsToPackage{dvipsnames,svgnames,x11names}{xcolor}
%
\documentclass[
  11pt,
  a4paper,
]{article}
\usepackage{amsmath,amssymb}
\usepackage{setspace}
\usepackage{iftex}
\ifPDFTeX
  \usepackage[T1]{fontenc}
  \usepackage[utf8]{inputenc}
  \usepackage{textcomp} % provide euro and other symbols
\else % if luatex or xetex
  \usepackage{unicode-math} % this also loads fontspec
  \defaultfontfeatures{Scale=MatchLowercase}
  \defaultfontfeatures[\rmfamily]{Ligatures=TeX,Scale=1}
\fi
\usepackage[]{mathpazo}
\ifPDFTeX\else
  % xetex/luatex font selection
\fi
% Use upquote if available, for straight quotes in verbatim environments
\IfFileExists{upquote.sty}{\usepackage{upquote}}{}
\IfFileExists{microtype.sty}{% use microtype if available
  \usepackage[]{microtype}
  \UseMicrotypeSet[protrusion]{basicmath} % disable protrusion for tt fonts
}{}
\makeatletter
\@ifundefined{KOMAClassName}{% if non-KOMA class
  \IfFileExists{parskip.sty}{%
    \usepackage{parskip}
  }{% else
    \setlength{\parindent}{0pt}
    \setlength{\parskip}{6pt plus 2pt minus 1pt}}
}{% if KOMA class
  \KOMAoptions{parskip=half}}
\makeatother
\usepackage{xcolor}
\usepackage[margin=1in]{geometry}
\usepackage{longtable,booktabs,array}
\usepackage{calc} % for calculating minipage widths
% Correct order of tables after \paragraph or \subparagraph
\usepackage{etoolbox}
\makeatletter
\patchcmd\longtable{\par}{\if@noskipsec\mbox{}\fi\par}{}{}
\makeatother
% Allow footnotes in longtable head/foot
\IfFileExists{footnotehyper.sty}{\usepackage{footnotehyper}}{\usepackage{footnote}}
\makesavenoteenv{longtable}
\usepackage{graphicx}
\makeatletter
\def\maxwidth{\ifdim\Gin@nat@width>\linewidth\linewidth\else\Gin@nat@width\fi}
\def\maxheight{\ifdim\Gin@nat@height>\textheight\textheight\else\Gin@nat@height\fi}
\makeatother
% Scale images if necessary, so that they will not overflow the page
% margins by default, and it is still possible to overwrite the defaults
% using explicit options in \includegraphics[width, height, ...]{}
\setkeys{Gin}{width=\maxwidth,height=\maxheight,keepaspectratio}
% Set default figure placement to htbp
\makeatletter
\def\fps@figure{htbp}
\makeatother
\setlength{\emergencystretch}{3em} % prevent overfull lines
\providecommand{\tightlist}{%
  \setlength{\itemsep}{0pt}\setlength{\parskip}{0pt}}
\setcounter{secnumdepth}{-\maxdimen} % remove section numbering
\pagestyle{plain}
% definitions for citeproc citations
\NewDocumentCommand\citeproctext{}{}
\NewDocumentCommand\citeproc{mm}{%
  \begingroup\def\citeproctext{#2}\cite{#1}\endgroup}
\makeatletter
 % allow citations to break across lines
 \let\@cite@ofmt\@firstofone
 % avoid brackets around text for \cite:
 \def\@biblabel#1{}
 \def\@cite#1#2{{#1\if@tempswa , #2\fi}}
\makeatother
\newlength{\cslhangindent}
\setlength{\cslhangindent}{1.5em}
\newlength{\csllabelwidth}
\setlength{\csllabelwidth}{3em}
\newenvironment{CSLReferences}[2] % #1 hanging-indent, #2 entry-spacing
 {\begin{list}{}{%
  \setlength{\itemindent}{0pt}
  \setlength{\leftmargin}{0pt}
  \setlength{\parsep}{0pt}
  % turn on hanging indent if param 1 is 1
  \ifodd #1
   \setlength{\leftmargin}{\cslhangindent}
   \setlength{\itemindent}{-1\cslhangindent}
  \fi
  % set entry spacing
  \setlength{\itemsep}{#2\baselineskip}}}
 {\end{list}}
\usepackage{calc}
\newcommand{\CSLBlock}[1]{\hfill\break\parbox[t]{\linewidth}{\strut\ignorespaces#1\strut}}
\newcommand{\CSLLeftMargin}[1]{\parbox[t]{\csllabelwidth}{\strut#1\strut}}
\newcommand{\CSLRightInline}[1]{\parbox[t]{\linewidth - \csllabelwidth}{\strut#1\strut}}
\newcommand{\CSLIndent}[1]{\hspace{\cslhangindent}#1}
\usepackage{lineno} % add 
\usepackage{amsmath} % add 
\usepackage{float} % place figures
\linenumbers % turns line numbering on 

% Allowing for landscape pages
\usepackage{lscape}
\newcommand{\blandscape}{\begin{landscape}}
\newcommand{\elandscape}{\end{landscape}}

% Left justification of the text: see https://www.sharelatex.com/learn/Text_alignment
% \usepackage[document]{ragged2e} % already in the latex template
\newcommand{\bleft}{\begin{flushleft}}
\newcommand{\eleft}{\end{flushleft}}


% Add Supplementary Tables and Figures
% Code from https://stackoverflow.com/a/51337664
\newcommand{\beginsupplement}{
  \setcounter{table}{0}  
  \renewcommand{\thetable}{S\arabic{table}}
  \setcounter{figure}{0} 
  \renewcommand{\thefigure}{S\arabic{figure}}
}



\ifLuaTeX
  \usepackage{selnolig}  % disable illegal ligatures
\fi
\usepackage{bookmark}
\IfFileExists{xurl.sty}{\usepackage{xurl}}{} % add URL line breaks if available
\urlstyle{same}
\hypersetup{
  pdftitle={Estimating invasive rodent abundance using removal data and hierarchical models},
  pdfauthor={Olivier Gimenez1*},
  colorlinks=true,
  linkcolor={RoyalBlue},
  filecolor={Maroon},
  citecolor={Blue},
  urlcolor={RoyalBlue},
  pdfcreator={LaTeX via pandoc}}

\title{Estimating invasive rodent abundance using removal data and hierarchical models}
\author{Olivier Gimenez\textsuperscript{1}*}
\date{2024-12-29}

\begin{document}
\maketitle

\setstretch{2}
\linenumbers

\small

\textsuperscript{1} CEFE, Univ Montpellier, CNRS, EPHE, IRD, Montpellier, France

\texttt{*} Corresponding author: \href{mailto:olivier.gimenez@cefe.cnrs.fr}{\nolinkurl{olivier.gimenez@cefe.cnrs.fr}}

\normalsize

\vspace{1cm}
\hrule

Invasive rodents pose significant ecological, economic, and public health challenges. Robust methods are needed for estimating population abundance to guide effective management. Traditional methods such as capture-recapture are often impractical for invasive species due to ethical and logistical constraints. Here, I showcase the application of hierarchical multinomial N-mixture models for estimating the abundance of invasive rodents using removal data. First, I performed a simulation study which demonstrated minimal bias in abundance estimates across a range of sampling scenarios. Second, I analyzed removal data for two invasive rodent species: coypus (\emph{Myocastor coypus}) in France and muskrats (\emph{Ondatra zibethicus}) in the Netherlands. Using hierarchical multinomial N-mixture models, I examined the effects of temperature on abundance while accounting for imperfect and time-varying capture probabilities. I also showed how to accommodate spatial variability using random effects, and quantified uncertainty in parameter estimates. Overall, I hope to demonstrate the flexibility and utility of hierarchical models in invasive species management. I provide reproducible code and data to encourage broader adoption of multinomial N-mixture models.

\vspace{3mm}
\hrule
\vspace{5mm}

\emph{Keywords}: Invasive species, Multinomial N-mixture, Population size, Statistical ecology

\newpage

\section{Introduction}\label{introduction}

Invasive species are a significant global issue, with wide-ranging impacts on ecosystems, economies, and public health (\citeproc{ref-Petr2020}{Pyšek et al. 2020}, \citeproc{ref-Roy2024}{Roy et al. 2024}). Among these, the financial, epidemiological, social, and ecological costs associated with invasive rodents are substantial, as they damage infrastructures, degrade agricultural systems, and act as reservoirs for zoonotic diseases (\citeproc{ref-Diagne2023}{Diagne et al. 2023}).

Effective management of invasive species requires the estimation of population abundance for guiding control efforts and evaluating the success of eradication or regulation programs (\citeproc{ref-Williams2002}{Williams et al. 2002}, \citeproc{ref-Thompson2021}{Thompson et al. 2021}). However, the challenge in estimating animal abundance is that individuals are not always observed even when present due to imperfect detection (\citeproc{ref-Borchers2002}{Borchers et al. 2002}, \citeproc{ref-Seber2023}{Seber and Schofield 2023}). Ignoring imperfect detection leads to biased estimates of population abundance (\citeproc{ref-Kery2008}{Kéry and Schmidt 2008}). To account for imperfect detection, capture-recapture methods are usually used to correct observed counts (\citeproc{ref-Mccrea2015}{McCrea and Morgan 2015}). Yet, for invasive species, capture-recapture is often impractical, as ethical and management concerns typically prevent the release of captured animals.

An alternative approach involves the use of removal methods (\citeproc{ref-Rodriguez2021}{Rodriguez de Rivera and McCrea 2021}) in which individuals are captured and permanently removed from the study area during successive sampling occasions. This process leads to a decrease in the expected number of captures by a consistent proportion over time (rather than by a fixed amount decline), which informs on the total abundance as the initial population determines how quickly the number of individuals available for capture diminishes.

While standard removal methods are well-established (\citeproc{ref-Moran1951}{Moran 1951}, \citeproc{ref-Zippin1956}{Zippin 1956}, \citeproc{ref-Zippin1958}{1958}, \citeproc{ref-Rodriguez2021}{Rodriguez de Rivera and McCrea 2021}), recent advances in population ecology remain underutilized in the context of invasive species. Hierarchical models, in particular, have gained traction (\citeproc{ref-RD2008}{Royle and Dorazio 2008}, \citeproc{ref-KR2015}{Kéry and Royle 2015}) due to their ability to: (i) explicitly separate biological processes of interest (e.g., population dynamics) from observation processes (e.g., imperfect detection), thus enabling more accurate modeling; (ii) incorporate environmental, spatial, or temporal covariates at multiple levels, allowing exploration of how various factors influence ecological processes; and (iii) share information across groups by modeling parameters hierarchically with random effects, which improves estimates for groups with fewer data.

In this paper, I showcase the application of a hierarchical formulation of removal models, the multinomial N-mixture model (\citeproc{ref-Dorazio2005}{Dorazio et al. 2005}), to estimate the abundance of rodents in Europe. In this study, I focus on the coypu (\emph{Myocastor coypus}) in France and the muskrat (\emph{Ondatra zibethicus}) in the Netherlands. Both species are semi-aquatic rodents introduced to Europe in the early 20th century following escapes or releases from fur farms. The coypu, native to South America, has formed widespread invasive populations in France (\citeproc{ref-Bonnet2023}{Bonnet et al. 2023}), where it causes significant damage to infrastructure and crops. Additionally, it serves as a healthy carrier of leptospirosis, a zoonotic disease with potentially serious consequences. Similarly, the muskrat, native to North America, has established extensive populations in the Netherlands. By burrowing into riverbanks, dykes, and dams, muskrats compromise the integrity of these structures, posing a threat to public safety (\citeproc{ref-vanloon2017}{Loon et al. 2017}).

Using removal data, I demonstrate the application of the multinomial N-mixture model to estimate the abundance of rodent populations. First, I conduct a simulation study to evaluate the model's performance under varying numbers of sampling sites and sampling occasions. Second, I present a case study on a coypu population in France to illustrate the hierarchical structure of the multinomial N-mixture model, demonstrating how covariates can be incorporated to account for variations in abundance and capture probabilities. Third, I use a case study on muskrats in the Netherlands to demonstrate the integration of random effects within the model. To facilitate reproducibility, I provide the accompanying code and data, aiming to promote the broader adoption of removal models in the study of biological invasions.

\section{Methods}\label{methods}

\subsection{Multinomial N-mixture model}\label{multinomial-n-mixture-model}

Think of a dice with six sides. The dice has a 1 in 6 chance of landing on face 1, the same for face 2, and so on. If I roll the dice 30 times, I would expect, on average over many repetitions of this experiment, to get face 1 five times, face 2 five times, and so on. You can test this in \texttt{R} by running the command \texttt{rmultinom(n\ =\ 1,\ size\ =\ 30,\ prob\ =\ c(1/6,\ 1/6,\ 1/6,\ 1/6,\ 1/6,\ 1/6))} repeatedly. In this experiment, \(y_1\), the number of 1s, \(y_2\), the number of 2s, \(\ldots\), and \(y_6\), the number of 6s, follows a multinomial distribution with parameters the number of rolls (30) and probabilities \((1/6, 1/6, ..., 1/6)\).

Now think of a removal campaign conducted over 3 months. We record the number of rodents \(y_1\) captured in month 1, \(y_2\) in month 2, \(y_3\) in month 3, and let \(y_4\) represent the number of rodents never captured. Let \(p\) be the probability of capturing a rodent in a given month. The probability of capturing a rodent in the first month is \(\pi_1 = p\). The probability of capturing a rodent in the second month is \(\pi_2 = (1-p)p\) the probability of not capturing it in the first month \((1 - p)\) multiplied by the probability of capturing it in the second month \(p\). The probability of capturing a rodent in the third month is \(\pi_3 = (1-p)(1-p)p\), the probability of not capturing it in the first and second months, \((1 - p)(1 - p)\), multiplied by the probability of capturing it in the third month, \(p\). Finally, the probability of never being captured is \(\pi_4 = 1 - (\pi_1 + \pi_2 + \pi_3)\) the complement of the probability of being captured in the first, second, or third month. If we assume that \(N\) represents the abundance, then we have that the vector of counts \((y_1, y_2, y_3, y_4)\) follows a multinomial distribution with parameters \(N\) and probabilities \((\pi_1,\pi_2,\pi_3,\pi_4)\). This is the observation process. In general, we assume that \(N\) follows a Poisson distribution with parameter the expected number of rodents denoted \(\lambda\). This is the state or ecological process. And there you have it, the multinomial N-mixture model for a removal experiment, which is similar to throwing a dice \(N\) times and the \(\pi's\) give the probabilities that I get a given face of that dice. In general, we monitor rodents in several populations or sites \(i = 1,\ldots,S\) and we need to estimate local abundance \(N_i\). To do so, Dorazio et al. (\citeproc{ref-Dorazio2005}{2005}) extended multinomial N-mixture models to account for spatial variation in abundance and/or capture, and showed that abundance estimates had similar or better precision than those obtained from analyzing removal data for each population or site separately.

Parameters \(N\), \(p\), and \(\lambda\) are unknown and need to be estimated. In a frequentist framework, marginalization is performed by summing over all possible values of \(N\) (\citeproc{ref-Dorazio2005}{Dorazio et al. 2005}). In a Bayesian framework, all these parameters are estimated directly, which simplifies the process (\citeproc{ref-RD2006}{Royle and Dorazio 2006}). Both parameters, \(\lambda\) and \(p\), can be modeled as functions of explanatory spatial and temporal variables, in the spirit of generalized linear models, and Poisson or logistic regressions for example. I warmly recommend chapter 7 in Kéry and Royle (\citeproc{ref-KR2015}{2015}) for a detailed description of the multinomial mixture model.

\section{Simulation study}\label{simulation-study}

I conducted a simulation study to evaluate the model's performance by examining parameter bias under varying numbers of sampling sites and sampling occasions. I simulated removal data over 1, 5, 10 and 50 sites using a Poisson distribution with expected number of animals \(\lambda\) between 10 and 100 (20 values) for the ecological process. I simulated the observation process with a capture probability \(p\) varying between 0.3 and 0.9 (20 values) across 3, 5 and 10 occasions per site. In total, I considered 4800 scenarios. I fitted the multinomial N-mixture model to the simulated data within the frequentist framework using function \texttt{multinomPois()} in the \texttt{R} package \texttt{unmarked} (\citeproc{ref-Kellner_2023}{Kellner et al. 2023}), and I repeated this procedure 500 times. Eventually, I calculated the relative bias for each parameter.

\section{Case studies}\label{case-studies}

In this section, I analyzed removal data from two rodent species: coypus in France and muskrats in the Netherlands. With these case studies, I aimed at illustrating specific features of hierarchical multinomial N-mixture models. For both species, I explored the potential effect of temperature on abundance (e.g., \citeproc{ref-Gosling1981}{Gosling 1981}, \citeproc{ref-Simpson1993}{Simpson and Boutin 1993}). A comprehensive analysis of the ecological factors influencing population dynamics was beyond the scope of this work and will be addressed in future studies.

\subsection{Coypus in France}\label{coypus-in-france}

Removal data on coypus were collected from annual control operations conducted since 2015 in several cities within the Hérault department, located in the Occitanie region of southern France. These operations are carried out year-round, with the exception of July and August. Coypus are trapped using cages by a network of volunteers coordinated by the Syndicat Mixte du Bassin de l'Or and the Fédération Départementale des Chasseurs de l'Hérault. For this study, I focus on data from 2022, specifically from sampling occasions in February, March, and April. The data, covering \(S = 6\) cities, are summarized in Table \ref{tab:coypus}. I fitted a model where the expected number of coypus was modeled as a function of temperature, while the capture probability was allowed to vary by month.

\begin{table}[ht]
\centering
\begin{tabular}{lcccc}
\hline
\textbf{} & \textbf{Removed in} & \textbf{Removed in} & \textbf{Removed in} & \textbf{Averaged} \\
\textbf{City} & \textbf{February} & \textbf{March} & \textbf{April} & \textbf{temperature}\\
\hline
Candillargues & 18  & 12    & 38 & 9.5 \\
Lansargues    & 15  & 17    & 75 & 8.8 \\
Mauguio       & 20  & 9 & 6 & 9.2 \\
Saint-Nazaire-de-Pézan & 169    & 41    & 15 & 9.3 \\
Saint-Just    & 85  & 61    & 77 & 9.2 \\
Valergues     & 0 & 1   & 3 & 9.4 \\
\hline
\end{tabular}
\caption{Number of invasive coypus removed monthly and the average 3-month temperature across several cities in the Hérault department, France, in 2022.}
\label{tab:coypus}
\end{table}

\subsection{Muskrats in the Netherlands}\label{muskrats-in-the-netherlands}

Removal data on muskrats in the Netherlands were collected by professional trappers. The data were registered in atlas blocks (5 x 5 km) per periods of four weeks. For this study, I focus on data from 2014, specifically from sampling occasions in January, February, and March. The data were made available through the LIFE MICA project (\citeproc{ref-Cartuyvels2024}{Cartuyvels et al. 2024}) and can be freely downloaded from \url{https://www.gbif.org/dataset/7d75109d-a6cb-4e90-89d0-79d08577c580}. The data, covering \(S = 342\) cities, are presented in Figure \ref{fig:datsmuskrats}. I fitted the same model as for the coypus data, except that I added a site random effect on abundance to accommodate the spatial variation that was not explained by temperature.

\begin{figure}[H]

{\centering \includegraphics[width=0.7\linewidth]{dat_muskrats} 

}

\caption{Total number of invasive muskrats removed over the period January-February-March (top panel), and the average 3-month temperature (bottom panel) across the Netherlands in 2014.}\label{fig:datsmuskrats}
\end{figure}

\subsection{Implementation}\label{implementation}

For all analyses, I used the statistical language \texttt{R} (\citeproc{ref-R_Core_Team_2024}{R Core Team 2024}). I used the \texttt{tidyverse} (\citeproc{ref-Wickham_2019}{Wickham et al. 2019}) suite of packages for data manipulation and visualization, \texttt{sf} (\citeproc{ref-Pebesma_2023}{Pebesma and Bivand 2023}) for dealing with spatial data and \texttt{krigR} (\citeproc{ref-Kusch_2022}{Kusch and Davy 2022}) to get temperature data. For the two case studies, I fitted models within a Bayesian framework using Markov chain Monte Carlo (MCMC) algorithms. I used both the \texttt{NIMBLE} (\citeproc{ref-de_Valpine_2017}{de Valpine et al. 2017}) and the \texttt{ubms} (\citeproc{ref-Kellner_2021}{Kellner et al. 2021}) packages. The former offers high flexibility, enabling users to define custom likelihoods, though it requires manual coding, while the latter features simpler syntax with pre-built multinomial N-mixture models, albeit limited to a Poisson distribution for abundance. I specified weakly informative priors for all parameters, specifically normal distributions with mean 0 and standard deviation 1.5 for regression parameters, and a uniform distribution for the standard deviation of the random effects. I ran two chains for a total of 200,000 iterations with a burn-in of 20,000 iterations. I summarized posterior distributions with posterior mean and 95\% credible intervals. I assessed model convergence using R-hat values (\textless{} 1.1), effective sample size (\textgreater{} 100), and visual inspection of the trace plots.

\section{Results and discussion}\label{results-and-discussion}

The results of the simulation study are presented in Figure \ref{fig:bias}. Overall, the analysis revealed minimal bias, with the exception of one site (first row) that showed a notable deviation.

\begin{figure}[H]

{\centering \includegraphics[width=0.98\linewidth]{heatmap_bias} 

}

\caption{Relative bias in abundance estimates from a multinomial N-mixture model with constant parameters. Capture probabilities (X-axis) range from 0.3 to 0.9, while abundance (Y-axis) varies between 10 and 100 individuals. Scenarios consider 3, 5, and 10 capture occasions (columns) and 1, 5, 10, and 50 sites (rows). Results are based on 500 simulations.}\label{fig:bias}
\end{figure}

Increasing the number of sites to 10 significantly reduced this bias, and no bias was observed with 50 sites, supporting the recommendation by (\citeproc{ref-Dorazio2005}{Dorazio et al. 2005}) to analyze data jointly rather than separately. These findings align with previous simulation studies, which also reported minimal bias in removal models (e.g., \citeproc{ref-Womack2019}{Womack-Bulliner et al. 2019}). To enhance reproducibility, I provide the code for the simulation study in the Supplementary Material. This resource can be adapted for various purposes, such as conducting custom simulation studies or designing removal protocols and conducting power analyses.

In the coypus case study, temperature was found to have a negative effect on abundance, with a slope estimate of -0.14 (-0.22, -0.07). Capture probabilities were estimated at 0.43 (0.29, 0.49) in February, 0.35 (0.19, 0.44) in March and 0.84 (0.35, 1.00) in April. The posterior distributions of abundance across the different sites are presented in Figure \ref{fig:pdcoypus}.

\begin{figure}[H]

{\centering \includegraphics[width=0.98\linewidth]{hist_coypus} 

}

\caption{Posterior density plots for coypu abundance across several cities in the Hérault department, France, in 2022. The vertical shaded line indicates the posterior mean abundance, accompanied by its credible interval. See Table 1 for the raw data.}\label{fig:pdcoypus}
\end{figure}

A key assumption of the multinomial N-mixture model is that abundance follows a Poisson distribution, which assumes the mean and variance are equal. However, this assumption does not appear valid based on the estimated abundance. Fortunately, this limitation can be addressed by relaxing the Poisson assumption. A straightforward approach is to use a negative binomial distribution to account for overdispersion. This adjustment can be implemented in both \texttt{NIMBLE} and \texttt{unmarked} but is not currently supported by \texttt{ubms}. I demonstrate how to fit such a model to the coypus data in the Supplementary Material. Interestingly, under the negative binomial model, the effect of temperature on abundance was no longer significant, with a slope estimate of -0.27 (-1.29, 0.60).

In the muskrats case study, temperature was found to have a negative effect on abundance, with a slope estimate of -0.48 (-0.70, -0.26). The standard deviation of the site random effect was estimated at 1.62 (1.46, 1.79). Capture probabilities were estimated at 0.12 (0.05, 0.25) in January, 0.25 (0.15, 0.35) in February and 0.60 (0.53, 0.75) in March. Estimated abundance across sites after removal is presented in Figure \ref{fig:muskrats}.

\begin{figure}[H]

{\centering \includegraphics[width=0.85\linewidth]{muskrats} 

}

\caption{Posterior mean estimates of the number of muskrats remaining after removal in the Netherlands in 2022 (top panel) and the corresponding coefficient of variation (bottom panel). See Figure 1 for the raw data.}\label{fig:muskrats}
\end{figure}

An important feature of multinomial N-mixture models is their ability to quantify uncertainty, which is often overlooked in spatial analyses. Here, I provide the coefficient of variation to represent the uncertainty surrounding abundance estimates. This metric can help identify specific areas where increased sampling effort might be beneficial to improve estimate precision.

Another key assumption underlying the proper use of multinomial N-mixture models is that of population closure, which assumes no births, deaths, immigration, or emigration occur during the trapping period. A straightforward approach to relax this assumption is to fit multiple years of data (a.k.a. stacking the data) into a standard multinomial N-mixture model. In this approach, year-site combinations are treated as separate sites, and year is included as a site covariate or random effect in the model. In the muskrats example, this method can be used to evaluate a temporal effect on the relationship between temperature and abundance. Assuming an increase in temperature over time, one might predict a decoupling or weakening of the relationship between abundance and temperature. To test this, I conducted an additional analysis of the muskrat dataset spanning the 1987--2014 period, modeling the slope of the temperature-abundance relationship as a linear function of time. The results revealed a negative trend in the slope, estimated at -0.040 (-0.042, -0.039), providing evidence to support the hypothesis of a temporal weakening of this relationship.

Several perspectives arise from this work. From a methodological standpoint, this study highlights the suitability of hierarchical models for capturing dependencies in space and time, which are common in ecological systems and removal experiments in particular. Two areas stand out as particularly worth exploring. First, I focused in this paper exclusively on static models under the assumption of population closure (see however previous section). When the mechanisms underlying population dynamics---such as survival, recruitment, or dispersal---are of interest, multinomial N-mixture models can be extended to open populations (\citeproc{ref-Matechou2016}{Matechou et al. 2016}, \citeproc{ref-Link2018}{Link et al. 2018}, \citeproc{ref-Zhou2019}{Zhou et al. 2019}). These extended models can be implemented using \texttt{unmarked} or \texttt{NIMBLE}, though they are not yet available in \texttt{ubms}.

A second area of investigation concerns the spatial dimension of multinomial N-mixture models, particularly the assumption of independence among removal sites (i.e., that removals at one site do not influence those at another). One possible solution is to include site random effects, as demonstrated in the muskrat case study. To better address spatial autocorrelation, restricted spatial regression (RSR) can also be employed (\citeproc{ref-Johnson2013}{Johnson et al. 2013}, \citeproc{ref-Broms2014}{Broms et al. 2014}) to impose a structure where spatially adjacent sites are modeled to have correlated random effects, effectively accounting for spatial autocorrelation. RSR models are advantageous because their random effects are constructed to be uncorrelated with fixed covariates, avoiding potential confounding issues, and they are computationally efficient. These models are easy to fit using \texttt{ubms} and can also be implemented in \texttt{NIMBLE} (\citeproc{ref-Cook2022}{Cook et al. 2022}), although they are not currently supported by \texttt{unmarked}. A promising extension would involve adapting the covariance structure in these models to account for stream networks (\citeproc{ref-Gimenez2024}{Gimenez 2024}, \citeproc{ref-Lu2024}{Lu et al. 2024}), which is particularly relevant for semi-aquatic rodents.

From an ecological perspective, the main contribution of this paper is to demonstrate the estimation of abundance for rodent populations in Europe. The European Union Regulation No.~1143/2014 was established to mitigate the negative impacts of invasive alien species on biodiversity. This regulation mandates measures to prevent the introduction of invasive alien species and manage their established populations. In this study, I focused on two species listed as species of Union Concern under the regulation, which requires member states to implement appropriate management actions.

In this context, although it is straightforward to calculate the number of coypus or muskrats remaining after removal campaigns (including associated uncertainty; see Figure \ref{fig:muskrats}), abundance estimates realize their full potential only when integrated into a decision-making framework. Such a framework allows for evaluating the efficiency of eradication or control efforts, optimizing the allocation of limited resources, and accounting for uncertainty in the management of invasive species (\citeproc{ref-Shea2002}{Shea et al. 2002}, \citeproc{ref-Adams2024}{Yackel Adams et al. 2024}).

\section{Acknowledgments}\label{acknowledgments}

I would like to warmly thank Nathalie Vazzoler-Antoine and Tanguy Lebrun for sharing the coypus data, and the team LIFE MICA for sharing the muskrats data. This work was funded by a grant from the University of Montpellier through its ExposUM institute.

\subsection{Data availability statement}\label{data-availability-statement}

Data and code are available at \href{https://github.com/oliviergimenez/counting-rodents}{https://github.com/oliviergimenez/counting-rodents}.

\section{References}\label{references}

\phantomsection\label{refs}
\begin{CSLReferences}{1}{0}
\bibitem[\citeproctext]{ref-Bonnet2023}
Bonnet, M., G. Guédon, S. Bertolino, C. Harmange, A. Pagano, D. Picard, and O. Pays. 2023. Improving the management of aquatic invasive alien rodents in france: Appraisal and recommended actions. Management of Biological Invasions 14:625--640.

\bibitem[\citeproctext]{ref-Borchers2002}
Borchers, D. L., S. T. Buckland, and Z. W. 2002. Estimating animal abundance: Closed populations. Springer-Verlag, London.

\bibitem[\citeproctext]{ref-Broms2014}
Broms, K. M., D. S. Johnson, R. Altwegg, and L. L. Conquest. 2014. Spatial occupancy models applied to atlas data show southern ground hornbills strongly depend on protected areas. Ecological Applications 24:363--374.

\bibitem[\citeproctext]{ref-Cartuyvels2024}
Cartuyvels, E., F. Huysentruyt, D. Brosens, P. Desmet, S. Devisscher, H. Fritz, L. Fromme, F. Gethöffer, C. Maistrelli, D. Moerkens, N. Noé, D. Slootmaekers, and T. Adriaens. 2024. Dataflows in support of cross-border management of muskrat (ondatra zibethicus) and coypu (myocastor coypus): The LIFE MICA approach. Management of Biological Invasions 15:455--470.

\bibitem[\citeproctext]{ref-Cook2022}
Cook, J. D., D. M. Williams, W. F. Porter, and S. A. Christensen. 2022. Improved predictions and forecasts of chronic wasting disease occurrence using multiple mechanism dynamic occupancy modeling. The Journal of Wildlife Management 86:e22296.

\bibitem[\citeproctext]{ref-de_Valpine_2017}
de Valpine, P., D. Turek, C. Paciorek, C. Anderson-Bergman, D. Temple Lang, and R. Bodik. 2017. \href{https://doi.org/10.1080/10618600.2016.1172487}{Programming with models: Writing statistical algorithms for general model structures with {NIMBLE}}. Journal of Computational and Graphical Statistics 26:403--413.

\bibitem[\citeproctext]{ref-Diagne2023}
Diagne, C., L. Ballesteros-Mejia, R. Cuthbert, T. Bodey, J. Fantle-Lepczyk, E. Angulo, A. Bang, G. D. G, and F. Courchamp. 2023. Economic costs of invasive rodents worldwide: The tip of the iceberg. PeerJ 11:e14935.

\bibitem[\citeproctext]{ref-Dorazio2005}
Dorazio, R. M., H. L. Jelks, and F. Jordan. 2005. Improving removal-based estimates of abundance by sampling a population of spatially distinct subpopulations. Biometrics 61:1093--1101.

\bibitem[\citeproctext]{ref-Gimenez2024}
Gimenez, O. 2024. Spatial occupancy models for data collected on stream networks. Aquatic Conservation: Marine and Freshwater Ecosystems 34:e70013.

\bibitem[\citeproctext]{ref-Gosling1981}
Gosling, L. M. 1981. Climatic determinants of spring littering by feral coypus, myocastor coypus. Journal of Zoology 195:281--288.

\bibitem[\citeproctext]{ref-Johnson2013}
Johnson, D. S., P. B. Conn, M. B. Hooten, J. C. Ray, and B. A. Pond. 2013. Spatial occupancy models for large data sets. Ecology 94:801--808.

\bibitem[\citeproctext]{ref-Kellner_2021}
Kellner, K. F., N. L. Fowler, T. R. Petroelje, T. M. Kautz, D. E. Beyer, and J. L. Belant. 2021. \href{https://doi.org/10.1111/2041-210X.13777}{{ubms}: An {R} package for fitting hierarchical occupancy and n-mixture abundance models in a bayesian framework}. Methods in Ecology and Evolution 13:577--584.

\bibitem[\citeproctext]{ref-Kellner_2023}
Kellner, K. F., A. D. Smith, J. A. Royle, M. Kery, J. L. Belant, and R. B. Chandler. 2023. \href{https://www.jstatsoft.org/v43/i10/}{The {unmarked} {R} package: Twelve years of advances in occurrence and abundance modelling in ecology}. Methods in Ecology and Evolution 14:1408--1415.

\bibitem[\citeproctext]{ref-KR2015}
Kéry, M., and J. A. Royle. 2015. Applied hierarchical modeling in ecology: Analysis of distribution, abundance and species richness in r and BUGS. Academic Press, London, UK.

\bibitem[\citeproctext]{ref-Kery2008}
Kéry, M., and B. Schmidt. 2008. Imperfect detection and its consequences for monitoring in conservation. Community Ecology 9:207--216.

\bibitem[\citeproctext]{ref-Kusch_2022}
Kusch, E., and R. Davy. 2022. \href{https://doi.org/10.1088/1748-9326/ac48b3}{KrigR-a tool for downloading and statistically downscaling climate reanalysis data}. Environmental Research Letters 17.

\bibitem[\citeproctext]{ref-Link2018}
Link, W. A., S. J. Converse, A. A. Yackel Adams, and N. J. Hostetter. 2018. Analysis of population change and movement using robust design removal data. Journal of Agricultural, Biological and Environmental Statistics 23:463--477.

\bibitem[\citeproctext]{ref-vanloon2017}
Loon, E. E. van, D. Bos, C. J. van Hellenberg Hubar, and R. C. Ydenberg. 2017. A historical perspective on the effects of trapping and controlling the muskrat (ondatra zibethicus) in the netherlands. Pest Management Science 73:305--312.

\bibitem[\citeproctext]{ref-Lu2024}
Lu, X., Y. Kanno, G. P. Valentine, J. M. Rash, and M. B. Hooten. 2024. Using multi-scale spatial models of dendritic ecosystems to infer abundance of a stream salmonid. Journal of Applied Ecology 61:1703--1715.

\bibitem[\citeproctext]{ref-Matechou2016}
Matechou, E., R. S. McCrea, B. J. T. Morgan, D. J. Nash, and R. A. Griffiths. 2016. {Open models for removal data}. The Annals of Applied Statistics 10:1572--1589.

\bibitem[\citeproctext]{ref-Mccrea2015}
McCrea, R. S., and B. J. T. Morgan. 2015. Analysis of capture-recapture data. Chapman \& Hall.

\bibitem[\citeproctext]{ref-Moran1951}
Moran, P. A. P. 1951. A mathematical theory of animal trapping. Biometrika 38:307--311.

\bibitem[\citeproctext]{ref-Pebesma_2023}
Pebesma, E., and R. Bivand. 2023. \href{https://doi.org/10.1201/9780429459016}{{Spatial Data Science: With applications in R}}. {Chapman and Hall/CRC}.

\bibitem[\citeproctext]{ref-Petr2020}
Pyšek, P., P. E. Hulme, D. Simberloff, S. Bacher, T. M. Blackburn, J. T. Carlton, W. Dawson, F. Essl, L. C. Foxcroft, P. Genovesi, J. M. Jeschke, I. Kühn, A. M. Liebhold, N. E. Mandrak, L. A. Meyerson, A. Pauchard, J. Pergl, H. E. Roy, H. Seebens, M. van Kleunen, M. Vilà, M. J. Wingfield, and D. M. Richardson. 2020. Scientists' warning on invasive alien species. Biological Reviews 95:1511--1534.

\bibitem[\citeproctext]{ref-R_Core_Team_2024}
R Core Team. 2024. \href{https://www.R-project.org/}{R: A language and environment for statistical computing}. R Foundation for Statistical Computing, Vienna, Austria.

\bibitem[\citeproctext]{ref-Rodriguez2021}
Rodriguez de Rivera, O., and R. McCrea. 2021. Removal modelling in ecology: A systematic review. Plos One 3:e0229965.

\bibitem[\citeproctext]{ref-Roy2024}
Roy, H. E., A. Pauchard, P. J. Stoett, T. Renard Truong, L. A. Meyerson, S. Bacher, B. S. Galil, P. E. Hulme, T. Ikeda, S. Kavileveettil, M. A. McGeoch, M. A. Nuñez, A. Ordonez, S. J. Rahlao, E. Schwindt, H. Seebens, A. W. Sheppard, V. Vandvik, A. Aleksanyan, M. Ansong, T. August, R. Blanchard, E. Brugnoli, J. K. Bukombe, B. Bwalya, C. Byun, M. Camacho-Cervantes, P. Cassey, M. L. Castillo, F. Courchamp, K. Dehnen-Schmutz, R. D. Zenni, C. Egawa, F. Essl, G. Fayvush, R. D. Fernandez, M. Fernandez, L. C. Foxcroft, P. Genovesi, Q. J. Groom, A. I. González, A. Helm, I. Herrera, A. J. Hiremath, P. L. Howard, C. Hui, M. Ikegami, E. Keskin, A. Koyama, S. Ksenofontov, B. Lenzner, T. Lipinskaya, J. L. Lockwood, D. C. Mangwa, A. F. Martinou, S. M. McDermott, C. L. Morales, J. Müllerová, N. A. Mungi, L. K. Munishi, H. Ojaveer, S. N. Pagad, N. P. K. T. S. Pallewatta, L. R. Peacock, E. Per, J. Pergl, C. Preda, P. Pyšek, R. K. Rai, A. Ricciardi, D. M. Richardson, S. Riley, B. J. Rono, E. Ryan-Colton, H. Saeedi, B. B. Shrestha, D. Simberloff, A. Tawake, E. Tricarico, S. Vanderhoeven, J. Vicente, M. Vilà, W. Wanzala, V. Werenkraut, O. L. F. Weyl, J. R. U. Wilson, R. O. Xavier, and S. R. Ziller. 2024. Curbing the major and growing threats from invasive alien species is urgent and achievable. Nature Ecology \& Evolution 8:1216--1223.

\bibitem[\citeproctext]{ref-RD2008}
Royle, J. A., and R. M. Dorazio. 2008. Hierarchical modeling and inference in ecology: The analysis of data from populations, metapopulations and communities. Academic Press. San Diego, California.

\bibitem[\citeproctext]{ref-RD2006}
Royle, J. andrew, and R. Dorazio. 2006. Hierarchical models of animal abundance and occurrence. Journal of Agricultural, Biological, and Environmental Statistics 11:249--263.

\bibitem[\citeproctext]{ref-Seber2023}
Seber, G. A. F., and M. R. Schofield. 2023. Estimating presence and abundance of closed populations. Springer.

\bibitem[\citeproctext]{ref-Shea2002}
Shea, K., H. P. Possingham, W. W. Murdoch, and R. Roush. 2002. Active adaptive management in insect pest and weed control: Intervention with a plan for learning. Ecological Applications 12:927--936.

\bibitem[\citeproctext]{ref-Simpson1993}
Simpson, M. R., and S. Boutin. 1993. Muskrat life history: A comparison of a northern and southern population. Ecography 16:5--10.

\bibitem[\citeproctext]{ref-Thompson2021}
Thompson, B. K., J. D. Olden, and S. J. Converse. 2021. Mechanistic invasive species management models and their application in conservation. Conservation Science and Practice 3:e533.

\bibitem[\citeproctext]{ref-Wickham_2019}
Wickham, H., M. Averick, J. Bryan, W. Chang, L. D. McGowan, R. François, G. Grolemund, A. Hayes, L. Henry, J. Hester, M. Kuhn, T. L. Pedersen, E. Miller, S. M. Bache, K. Müller, J. Ooms, D. Robinson, D. P. Seidel, V. Spinu, K. Takahashi, D. Vaughan, C. Wilke, K. Woo, and H. Yutani. 2019. \href{https://doi.org/10.21105/joss.01686}{Welcome to the {tidyverse}}. Journal of Open Source Software 4:1686.

\bibitem[\citeproctext]{ref-Williams2002}
Williams, B. K., J. D. Nichols, and M. J. Conroy. 2002. Analysis and management of animal populations. Academic Press, San Diego, California, USA.

\bibitem[\citeproctext]{ref-Womack2019}
Womack-Bulliner, K. M., S. K. Amelon, F. R. Thompson, and J. J. Lebrun. 2019. {Performance of Hierarchical Abundance Models on Simulated Bat Capture Data}. Acta Chiropterologica 20:465--474.

\bibitem[\citeproctext]{ref-Adams2024}
Yackel Adams, A. A., N. J. Hostetter, W. A. Link, and S. J. Converse. 2024. Identifying pareto-efficient eradication strategies for invasive populations. Conservation Letters 17:e13051.

\bibitem[\citeproctext]{ref-Zhou2019}
Zhou, M., R. S. McCrea, E. Matechou, D. J. Cole, and R. A. Griffiths. 2019. Removal models accounting for temporary emigration. Biometrics 75:24--35.

\bibitem[\citeproctext]{ref-Zippin1956}
Zippin, C. 1956. An evaluation of the removal method of estimating animal populations. Biometrics 12:163--189.

\bibitem[\citeproctext]{ref-Zippin1958}
Zippin, C. 1958. The removal method of population estimation. The Journal of Wildlife Management 22:82--90.

\end{CSLReferences}

\end{document}
